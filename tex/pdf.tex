\documentclass[CJK,aspectratio=43]{beamer}  %aspectratio控制页面的宽高比,16:9充满屏幕
\usepackage{xeCJK}
\usepackage{booktabs}
\usepackage{multirow}
\usepackage{amsmath}   % 数学公式排版
\usepackage{graphicx}  % 插入图片和图形
\usepackage{hyperref}  % 创建超链接和交互式文档
\usepackage{geometry}  % 设置页面布局和边距
\usepackage{babel}     % 多语言支持
\usepackage[utf8]{inputenc}   % 设置文本编码
\usepackage[T1]{fontenc}     % 设置字体编码
\usepackage{xcolor}    % 创建颜色和调整颜色
\usepackage{listings}  % 排版代码和程序清单
\usepackage{appendix}
\setmainfont[Mapping=tex-text]{Times New Roman}%--------------------------------英文衬线字体
\setsansfont[Mapping=tex-text]{Arial}%------------------------------------英文无衬线字体
\setmonofont[Mapping=tex-text]{Courier New}%-------------------------------------英文等宽字体
\newfontfamily\Arial{Arial}
% 默认的数学公式较难看,使用serif可以使得公式的样式变为常用latex的公式样式 
\usefonttheme[onlymath]{serif} 
\usetheme{Frankfurt} %对应网站的行
\usecolortheme{whale} %对应网站的列
\setbeamertemplate{headline}[frame number]
\expandafter\def\expandafter\insertshorttitle\expandafter{%
	\insertshorttitle\hfill%
	\insertframenumber\,/\,\inserttotalframenumber}
\definecolor{uibred}  {HTML}{db3f3d}
\definecolor{uibblue} {HTML}{4ea0b7}
\definecolor{uibgreen}{HTML}{789a5b}
\definecolor{uibgray} {HTML}{d0cac2}
\definecolor{uiblink} {HTML}{00769E}
\setbeamercolor{block body} {bg = uibgray}
\setbeamercolor{block title}{fg = white,bg = uibred}
\setbeamertemplate{caption}[numbered]
\title{基于收益预测和CVaR约束的投资组合优化}
\author{Jingsong Yuan}
\centering
\date{November 2023}
\begin{document}
	% 一般第一页显示PPT标题以及作者信息
	\begin{frame}
		\title{基于收益预测和CVaR约束的投资组合优化}
		\author{Jingsong Yuan} % 显示作者
		\date{2023.11}  % 显示日期
		\titlepage
	\end{frame}
	\begin{frame}{Outline}
		\tableofcontents[currentsection,hideallsubsections]
	\end{frame}
\section{研究背景}
\subsection{改进期望收益}
	\begin{frame}{改进期望收益}
			\begin{itemize}
				\item Mean-Variance存在缺陷,模型建立在资产收益率服从正态分布等诸多限制性假设之上;均值作为期望收益与实际市场不符;低通滤波;不适于短期投资;
			\begin{figure}
				\centering
				\includegraphics[width=0.9\linewidth]{"pic/distribution of random six"}
				\caption{Distribution}
				\label{fig:distribution-of-random-stock}
			\end{figure}
			\end{itemize}
	\end{frame}
\subsection{改进风险度量}
	\begin{frame}{改进风险度量}
		\begin{enumerate}
			\item 使用方差度量风险
			\begin{itemize}
				\item 对异常值敏感,极端风险事件,会显著影响方差计算;
				\item 只关注收益率分布的前两矩,忽略分布的尾部;
			\end{itemize}
			\item 使用CVaR来度量风险损失
			\begin{itemize}
				\item 对异常值不敏感
				\item 不依赖分布假设
				\item 考虑分布的尾部,更好处理异常值
			\end{itemize}
		\end{enumerate}
	\end{frame}
\section{理论模型}
\subsection{资产及收益率}
\begin{frame}{资产及收益率}
	\begin{itemize}
		\item $N$种资产
		\item \textbf{收益率矩阵} $N\times T$
		$$
		R=
		\begin{bmatrix}
			r_1^1 & r_1^2 & r_1^3 & ... & r_1^T\\
			r_2^1 & r_2^2 & r_2^3 & ... & r_2^T\\
			r_3^1 & r_3^2 & r_3^3 & ... & r_3^T\\
			... & ... & ... & ... & ...\\
			r_N^1 & r_N^2 & r_N^3 & ... & r_N^T\\
		\end{bmatrix}
		=
		\begin{bmatrix}
			R_1 & R_2 & R_3 &... & R_T
		\end{bmatrix}
		$$
		其中$R_t$为 $t$ 时期 $N$ 种资产的收益率矩阵;
		\item \textbf{权重矩阵} $N \times 1$
		$$
		w=\begin{bmatrix}
			w_1 & w_2 & w_3 & ... & w_N
		\end{bmatrix}^T
		$$
	\end{itemize}
	\end{frame}

\subsection{VaR及CVaR}
	\begin{frame}{VaR及CVaR}
	\begin{itemize}
		\item \textbf{价值损失函数}
		$$
		L(w,R)=-w^TR
		$$
		其中,$w\in W\subseteq R^N$,$W$为可行集;$R$为随机变量,密度函数为$P(R)$;
		\item $L(w,R)$是依赖于 $w$ 的随机变量,小于临界值 $\lambda$ 的概率为:
		$$
		\varphi(w,\lambda)=\int_{L(w,R)\leq\lambda}P(R)dR
		$$
		\item \textbf{置信水平}: $\alpha\in(0,1)$
		$$
		\text{VaR}:\lambda_{\alpha}(w)=\text{inf}\{\lambda\in R;\varphi(w,\lambda) \geq \alpha \}
		$$
		$$
		\text{CVaR}:\phi_{\alpha}(w)=E[L(w,R) \vert L(w,R) \geq VaR(w)]
		$$
		$$
		=\frac{1}{1-\alpha}\int_{L(w,R)\geq \lambda_{\alpha}(w)}L(w,R)P(R)dR
		$$
	\end{itemize}
	\end{frame}

\subsection{CVaR计算}
\begin{frame}{CVaR计算}
	\begin{itemize}
		\item 利用Rockafellar和Uryasev同时计算$\text{VaR}$和$\text{CVaR}$的功能函数$F_{\alpha}(w,R)$有
		$$
		F_{\alpha}(w,\lambda)=\lambda+\frac{1}{1-\alpha}\int_{R\in R^N}[L(w,R)-\lambda]^{+}P(R)dR
		$$
		其中,$\lambda$即$\text{VaR}$值,$[0,x]^+=\text{max}(0,x)$
		\item Rockafellar 和 Uryasev 提出,当$P(R)$的解析式未知时,可以使用基于历史数据的情景分析法,产生情景矩阵,将多重积分转化为求和运算;
		\item 假设有$m$种情况,可以取$N$种债券$m$个时期的收益率
		$$
		\widehat{F}_{\alpha}(w,R)=\lambda+\frac{1}{m(1-\alpha)}\sum_{t=1}^{m}[L(w,R_t)-\lambda]
		$$
	\end{itemize}
\end{frame}

\subsection{优化问题建模}
\begin{frame}{优化问题建模}
	\begin{itemize}
		\item 目标函数 \\
		$
		\text{min}_{w} \ \lambda+\frac{1}{m(1-\alpha)}\sum_{t=1}^{m}[-w^TR_t-\lambda]
		$
		\item 决策变量: $w$
		\item 约束条件 \\
		$
		\text{s.t.} \ w^TI=1
		$
	\end{itemize}
\end{frame}

\section{数据实验}
\begin{frame}{数据实验}
	\begin{enumerate}
		\item 历史数据
		\begin{itemize}
			\item 沪深300名单内股票及指数月度收益数据
			\item 时间范围2005.04(month85)至2021.12(month285)
		\end{itemize}
		\item 因子
		\begin{itemize}
			\item \textbf{财务因子:}利润能力、偿债能力、资产周转率、现金流;
			\item \textbf{成长因子:}营收增长、利润增长、现金流增长;
			\item \textbf{市场因子:}股价相关、市值因子、动量;
			\item \textbf{其他因子:}风险、交易量、评级、股东持股变化、技术指标;
		\end{itemize}
	\end{enumerate}
	\begin{table}[htbp]
		\centering
		\resizebox{\textwidth}{!}{%
		\begin{tabular}{|l|l|l|l|l|l|}
			\toprule
			\multicolumn{1}{|c|}{\textbf{财务因子}} & \multicolumn{1}{c|}{\textbf{成长因子}} & \multicolumn{1}{c|}{\textbf{市场因子}} & \multicolumn{1}{c|}{\textbf{风险因子}} & \multicolumn{1}{c|}{\textbf{技术指标}} & \multicolumn{1}{c|}{\textbf{其他因子}} \\
			\midrule
			ROE\_q & Sales\_G\_q & return\_1m & beta  & macd  & LN\_capital \\
			\midrule
			ROE\_ttm & Profit\_G\_q & return\_3m & std\_1m & dea   & HAlpha \\
			\midrule
			ROA\_q & OCF\_G\_q & return\_6m & std\_3m & dif   & grossprofitmargin\_q \\
			\bottomrule
		\end{tabular}%
		}
		\label{tab:factors}%
		\caption{Factors}
	\end{table}%
\end{frame}

\subsection{代码实现框架}
\begin{frame}{选股及预测}
	\begin{enumerate}
	\item 打分法选股
	\begin{itemize}
		\item 线性回归模型拟合训练集数据
		\item 分别使用回归系数、信息系数、因子协方差法为因子分配权重
		\item 根据不同股票的因子,得到分数较高的股票
	\end{itemize}
	\item 预测测试集收益率
	\begin{itemize}
		\item 使用PCA进行因子合成(Number of components based on Kaiser Criterion: 16)
		\begin{figure}
			\centering
			\includegraphics[width=0.6\linewidth]{pic/scree_plot}
			\caption{Scree Plot}
			\label{fig:screeplot}
		\end{figure}
		\item 拟合训练集的数据,预测测试集数据
	\end{itemize}
	\end{enumerate}
\end{frame}

\subsection{代码实现框架}
\begin{frame}{优化问题重构}
	\begin{enumerate}
		\item 决策变量为 $weights$
		\item 根据历史数据计算VaR
		\begin{itemize}
			\item 历史组合收益$ \text{Portfolio return} = Weights.T \times R $
			\item 置信水平为 $\alpha=99\%$
			\item 拟合组合收益的分布,取$1\%$处分位数为$VaR$值$=\lambda$
		\end{itemize}
		\item 加入预测数据计算CVaR(目标函数)
		\begin{itemize}
			\item 合并过去$40$个月历史数据,及预测的下$1$个月数据,形成新的数据集
			\item 代入$m$,$\alpha$,$L(w,R_t)=-Weights.T \times R_t$,$\lambda$到之前的解析式中
			$$
			\widehat{F}_{\alpha}(w,R)=\lambda+\frac{1}{m(1-\alpha)}\sum_{t=1}^{m}[L(w,R_t)-\lambda]
			$$
		\end{itemize}
	\end{enumerate}
\end{frame}

\subsection{代码实现框架}
\begin{frame}{Rolling Window}
	\begin{enumerate}
		\item 投资目标是短期投资
		\item 使用滚动训练的方法调整回归模型
		\begin{itemize}
			\item 训练集最少使用60个月的数据做训练
			\item 测试集是训练集之后的6个月
			\item 每6个月更新一次训练集及测试集
		\end{itemize}
		\item 调整每次选出来的股票
		\begin{itemize}
			\item 测试集中每个月都更新选股
		\end{itemize}
	\end{enumerate}
\end{frame}

\subsection{策略回测}
\begin{frame}{回测思路}
	\begin{enumerate}
		\item 三种因子分配权重方法
		\begin{itemize}
			\item 按回归系数归一化配比
			\item 按信息系数归一化配比
			\item 按协方差矩阵的逆做配比
		\end{itemize}
		\item 两种最优权重求解方法
		\begin{itemize}
			\item Mean-Variance模型最小方差组合的权重
			\item 使CVaR值最小的权重
		\end{itemize}
		\item 与沪深300指数的收益对比
		\begin{itemize}
			\item 分别将两个最优权重,带入当月的实际收益率,计算组合收益
			\item 将组合的收益与沪深300指数的收益对比
		\end{itemize}
	\end{enumerate}
\end{frame}

\subsection{策略回测}
\begin{frame}{策略收益表现-Coef}
	\begin{itemize}
		\item 10年期收益及累计收益; 总体暂时没有稳定跑赢指数; 
		\item MV策略总体优于CVaR
	\end{itemize}
	\begin{figure}
		\centering
		\includegraphics[width=0.8\linewidth]{pic/Coef}
		\caption{Return and Compound Value}
		\label{fig:coef}
	\end{figure}
\end{frame}

\subsection{策略回测}
\begin{frame}{策略回撤-Coef}
	\begin{itemize}
		\item 10年期收益及累计收益; 总体暂时没有稳定跑赢指数; 
		\item MV策略总体优于CVaR
	\end{itemize}
	\begin{figure}
		\centering
		\includegraphics[width=0.8\linewidth]{"pic/Coef_dramdown"}
		\caption{}
		\label{fig:coefdramdown}
	\end{figure}
\end{frame}

\subsection{策略回测}
\begin{frame}{策略收益表现-IC}
	\begin{itemize}
		\item 总体上基本可以跑赢指数; 
		\item MV策略与CVaR策略接近
	\end{itemize}
	\begin{figure}
		\centering
		\includegraphics[width=0.8\linewidth]{pic/IC}
		\caption{Return and Compound Value}
		\label{fig:ic}
	\end{figure}
\end{frame}

\subsection{策略回测}
\begin{frame}{策略回撤-IC}
	\begin{itemize}
		\item 10年期收益及累计收益; 总体暂时没有稳定跑赢指数; 
		\item MV策略总体优于CVaR
	\end{itemize}
	\begin{figure}
		\centering
		\includegraphics[width=0.8\linewidth]{"pic/IC_dramdown"}
		\caption{}
		\label{fig:icdramdown}
	\end{figure}
\end{frame}

\subsection{策略回测}
\begin{frame}{策略收益表现-Corr}
	\begin{itemize}
	\item 总体上大幅跑赢指数; 
	\item CVaR策略逐步优于MV策略
	\end{itemize}
	\begin{figure}
		\centering
		\includegraphics[width=0.8\linewidth]{pic/Corr}
		\caption{Return and Compound Value}
		\label{fig:corr}
	\end{figure}
\end{frame}

\subsection{策略回测}
\begin{frame}{策略回撤-Corr}
	\begin{itemize}
		\item 10年期收益及累计收益; 总体暂时没有稳定跑赢指数; 
		\item MV策略总体优于CVaR
	\end{itemize}
	\begin{figure}
		\centering
		\includegraphics[width=0.8\linewidth]{"pic/Corr_dramdown"}
		\caption{}
		\label{fig:corrdramdown}
	\end{figure}
\end{frame}

\section{不足与改进}
\begin{frame}{不足与改进}
	\begin{enumerate}
	\item 理论端
		\begin{itemize}
			\item 考虑交易成本
		\end{itemize}
		
	\item  预测端
		\begin{itemize}
			\item 尝试更多ML/DNN方法
		\end{itemize}
	\item  策略端
		\begin{itemize}
			\item 改进滚动及调仓周期
		\end{itemize}
	\item  数据端
		\begin{itemize}
			\item 使用更高频数据
		\end{itemize}
	\item  求解方法
		\begin{itemize}
			\item 松弛线性规划(LP)
			\item 次梯度算法(SA)
			\item 遗传算法(GA)
			\item 光滑化方法(SM)
		\end{itemize}
	\end{enumerate}
\end{frame}

\end{document}